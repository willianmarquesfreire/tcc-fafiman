\section*{\centering{RESUMO}}
Um resumo de trabalho de conclusão de curso é do tipo informativo e deve conter somente um parágrafo. A estrutura do resumo deve conter essencialmente os seguintes tópicos: apresentar inicialmente os objetivos do trabalho (o que foi feito?), a justificativa (porquê foi feito) e, finalmente, os resultados alcançados. O resumo deve informar ao leitor todas as informações importantes para o que o leitor possa entender o trabalho desenvolvido, quais foram as finalidades, a metodologia que o autor utilizou e os resultados obtidos. Deve conter frases curtas, porém completas (evitar estilo telegráfico); usar o tempo verbal no passado para os principais resultados e presente para comentários ou para salientar implicações significativas.  O resumo em português e inglês são obrigatórios e não devem passar de 200 palavras.

{\bf Palavras-chave:} $<$Primeira palavra$>$,  $<$segunda palavra$>$, $<$até 5 palavras$>$.
$<$ Obs.: as palavras-chave devem ser escolhidas com bastante rigor, pois devem representar adequadamente os principais temas abordados pela pesquisa.$>$


