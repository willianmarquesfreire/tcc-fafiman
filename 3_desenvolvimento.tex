\section{DESENVOLVIMENTO}

\subsection{Introdução ao Eureka}

Neste trabalho tem-se por objetivo o a pesquisa e desenvolvimento de uma estrutura IoT baseado no padrão de micro-serviços, e para isto precisa-se utilizar um dos princípios que foi citada na revisão bibliográfica que é o "Service Discovery" ou descoberta de serviços. Como o projeto será baseado em Java, inicialmente será criado um projeto e  utilizará maven para gerenciamento de dependências e o Eureka Client para descoberta de serviços. Para poder utilizar o maven deve-se criar um arquivo de configurações no diretório raíz do projeto chamado "pom.xml", e a documentação completa pode se encontrar no site de seus desenvolvedor. Posteriormente será incluso o seguinte groupId org.springframework.cloud e o artifactId spring-cloud-starter-eureka, e para mais informações pode ser encontrada na documentação oficial do Spring Cloud Netflix.

Quando um cliente se registra com o Eureka, ele fornece meta-dados sobre si, indicador de estado ou saúde, página inicial, dentre outros. Eureka recebe mensagens heartbeat (disponibilidade) de cada instância pertencente a um serviço. Se algum heartbeat falhar, a instância é removido do registro.
Para inicializar um projeto com Eureka Client, será utilizado algumas anotações Java fornecidas pelo Eureka descritas a seguir: @Configuration, para utilizar recursos do projeto Spring Config para facilitar configurações de projetos Spring baseado em Java, @ComponentScan para buscar componentes em pacotes java, @EnableAutoConfiguration para ativar a ComponentScandescritas a seguir: @EnableEurekaClient para ativar a descoberta de serviços do Eureka, @RestController para criar um controlador Rest (Representational State Transfer), @RequestMapping para mapear as rotas da aplicação.

\begin{verbatim}
@Configuration
@ComponentScan
@EnableAutoConfiguration
@EnableEurekaClient
@RestController
public class Application {

    @RequestMapping("/")
    public String home() {
        return "Hello world";
    }

    public static void main(String[] args) {
        new SpringApplicationBuilder(Application.class).web(true).run(args);
    }
}
\end{verbatim}

Para que possa surtir efeito na aplicação precisa fazer ajustes nas configurações do Eureka dentro do diretório resources da aplicação Java. Esta configuração é feita dentro de um arquivo Application.yml.

\begin{verbatim}
  Eureka
   cliente:
     ServiceUrl:
       DefaultZone: http: // localhost: 8761 / eureka / 
\end{verbatim}

Neste arquivo de configuração encontra-se uma peculariedade. O DefaultZone é a URL do serviço Eureka para qualquer cliente. O nome do aplicativo padrão (ID de serviço), o host e a porta podem ser acessadas respectivamente pelas variáveis de ambientes: \${spring.application.name} , \${spring.application.name} e \${server.port}.
A anotação Java @EnableEurekaClient faz com que o a aplicação corrente se registre no Eureka, para que assim possa localizar outros serviços.

\subsection{Status e Saúde do serviço}

Com a página de status e os indicadores de integridade de uma instância do Eureka é possível visualizar informações do serviço. Para acessar os indicadores de saúde deve-se configurar as rotas padrões de acesso a mesma. Por padrão, o eureka utiliza a conexão do cliente para determinar se um cliente está ativo. Caso não utilize o Discovery Client, não será propagado o status de verificação de integridade atual do serviço. Para funcionar corretamente os indicadores de saúde e status da aplicação, devem ser feitas as seguintes configurações:

\begin{verbatim}
eureka:
  instance:
    statusPageUrlPath: ${management.context-path}/info
    healthCheckUrlPath: ${management.context-path}/health
  client:
    healthcheck:
      enabled: true
\end{verbatim}

Para conseguir utilizar mais recursos e obter mais informações sobre o status da aplicação, a aplicação deve implementar seu próprio controle de integridade que se encontra no pacote com.netflix.appinfo.HealthCheckHandler

\subsection{Registrando um serviço}

