%%%%%%%%%%%%%%%%%%%%%%%%%%%%% TCC %%%%%%%%%%%%%%%%%%%%%%%%%%%%%%%%
%
% Template para TCC da Universidade Federal da Paraíba
%
% Autores: Elaine Soares elaineanita1@gmail.com
%          Rafael Brayner rafabrayner92@gmail.com
%          Roberto Júnior contato@robertojunior.net
%
% ShareLaTeX porting: Gustavo Sobral ghsobral@gmail.com
% 
% Revisão: Eudisley Anjos eudisley@ci.ufpb.br
%
% Sinta-se livre para melhorar e contribuir com esse projeto. 
%
%%%%%%%%%%%%%%%%%%%%%%%%%%%%%%%%%%%%%%%%%%%%%%%%%%%%%%%%%%%%%%%%%%%

\documentclass{tcc}

\begin{document}
\pagestyle{empty} %retira numeração da página
%Dados do TCC%
\author{Willian Marques Freire}
\title{IoT e integração com Micro-serviços}
\newcommand{\subtitulo}{Subtítulo}
\newcommand{\nomedocurso}{Ciência da Computação}
\newcommand{\titulobar}{Ciência da Computação}
\newcommand{\orientador}{Munif Gebara Júnior}
\newcommand{\profa}{Nome do Professor A}
\newcommand{\profb}{Nome do Professor B}
\newcommand{\profc}{Nome do Professor C}
\newcommand{\insta}{Instituicao do Professor A}
\newcommand{\instb}{Instituicao do Professor B}
\newcommand{\instc}{Instituicao do Professor C}
\newcommand{\coordenador}{Nome do Coordenador}
\newcommand{\departamento}{Nome do Departamento}

\begin{center}
\LARGE{\bf
Fundação Faculdade de Filosofia, Ciências e Letras de Mandaguari
}\\
\Large{\bf
    Curso de Ciência da Computação
}
\end{center}
\begin{figure}[H]
\vspace*{3cm}
\centering
\includegraphics[width=100mm]{imagens/logo2.jpg}
\vspace*{3cm}
\end{figure}


\begin{center}
\LARGE{\bf \thetitle}\\
\end{center}

\vspace{1em}

\vfill

\vspace{2in}

\begin{center}
\bf\theauthor
\vspace*{2cm}
\end{center}

\begin{center}
Mandaguari, \the\year
\end{center}
\afterpage{\blankpage \addtocounter{page}{1}} %addtocounter incrementa numero da pagina ja que blankpage nao entra no contador%

\newpage
\begin{center}
\theauthor
\end{center}
\vspace{3in}
\begin{center}
\LARGE{\thetitle}\\
\end{center}

\vspace{2in}

\begin{flushright}
Monografia apresentada ao curso \nomedocurso \\ do Centro de Informática, da Fundação Faculdade de Filosofia Ciencias e Letras de Mandaguari, \\ como requisito para a obtenção do grau de Bacharel em \titulobar
\\
\vspace{0.2in}

Orientador: \orientador


\end{flushright}

\vfill
\begin{center}
\MONTH de \the\year
\end{center}

\newpage

$ $
\vfill


\begin{flushright}
\fbox{\parbox[t][15em][c]{0.90\linewidth}{
\vspace{0.5in}

$\qquad$ Ficha catalográfica: elaborada pela biblioteca do CI. 
\vspace{0.15in}

$\qquad$ Será impressa no verso da folha de rosto e não deverá ser contada. 

$\qquad$ Se não houver biblioteca, deixar em branco.
}}
\end{flushright}

\newpage

\begin{figure}[H]
\centering
\includegraphics[width=100mm]{imagens/logo3.jpg}
\end{figure}

\begin{center}
CENTRO DE INFORMÁTICA \\
Fundação Faculdade de Filosofia, Ciências e Letras de Mandaguari
\end{center}

\vspace{0.05in}

Trabalho de Conclusão de Curso de \nomedocurso  intitulado \textit{\bf \em \thetitle} de autoria de \theauthor, aprovada pela banca examinadora constituída pelos seguintes professores: \\

\vspace{0.7in}

\hrule
\noindent Prof. Mr. \profa\\
\insta\\

\vspace{0.25in}

\hrule
\noindent Prof. \profb\\
\instb\\

\vspace{0.25in}

\hrule
\noindent Prof. \profc\\
\instc\\

\vspace{0.8in}

\hrule
\noindent Coordenador(a) do Departamento \departamento\\
\coordenador\\
CI/UFPB\\

\vfill

\begin{center}
Mandaguari, \today
\end{center}

\vspace{0.05in}

\begin{center}
\footnotesize{ Fundação Faculdade de Filosofia, Ciências e Letras de Mandaguari\\
Rua Rene Taccola, 152  Centro, Mandaguari, Paraná, Brasil CEP: 86975-000\\
Fone: +55 (44) 3233-1356}
\end{center}
\afterpage{\blankpage \addtocounter{page}{1}}
%página em branco%
\newpage
$ $
\vfill

\begin{flushright}
\em *** O sucesso é ir de fracasso em fracasso sem perder entusiasmo.\\
Winston Churchill ***
\end{flushright}

\afterpage{\blankpage \addtocounter{page}{1}}

\newpage

%Dedicatória%
\section*{\centering{DEDICATÓRIA}} 
A Deus, que nos criou e foi criativo nesta tarefa. Ao fôlego de vida que me tem sustentado na coragem de questionar o sentido da vida e propor soluções nas possíveis evoluções e revoluções que tem chegado à humanidade. A meu pai e mãe que me são meu motivo para viver e crescer. A meu orientador e professor, que não tem somente me mostrado um novo mundo tecnológico, mas também têm me ensinado a viver e crescer no mesmo. E a todos que me apoiam e motivam a acreditar em um mundo melhor.

\newpage

%Agradecimentos%
\section*{\centering{AGRADECIMENTOS}} 
***O agradecimento é opcional***

\newpage

%Resumo%
\section*{\centering{RESUMO}}
Um resumo de trabalho de conclusão de curso é do tipo informativo e deve conter somente um parágrafo. A estrutura do resumo deve conter essencialmente os seguintes tópicos: apresentar inicialmente os objetivos do trabalho (o que foi feito?), a justificativa (porquê foi feito) e, finalmente, os resultados alcançados. O resumo deve informar ao leitor todas as informações importantes para o que o leitor possa entender o trabalho desenvolvido, quais foram as finalidades, a metodologia que o autor utilizou e os resultados obtidos. Deve conter frases curtas, porém completas (evitar estilo telegráfico); usar o tempo verbal no passado para os principais resultados e presente para comentários ou para salientar implicações significativas.  O resumo em português e inglês são obrigatórios e não devem passar de 200 palavras.

{\bf Palavras-chave:} $<$Primeira palavra$>$,  $<$segunda palavra$>$, $<$até 5 palavras$>$.
$<$ Obs.: as palavras-chave devem ser escolhidas com bastante rigor, pois devem representar adequadamente os principais temas abordados pela pesquisa.$>$



\newpage

%Abstract%
\section*{\centering{ABSTRACT}} 
% $<$Resumo em Inglês - Write here the abstract of your work$>$
This paper aims to present the IoT \emph{(Internet of Things)} concepts and Micro-services, and to prove that it is possible to develop a structure that allows the integration between these technologies. One of the justifications for the development of this work, tends to the fact that several studies have already appeared on the mentioned subjects, and have benefited the applications developed, with high cohesion and low coupling when it comes to micro-services, and favoring the sharing of Information and integration into the global Internet network when it comes to IoT. Associating these two technologies that are directed to applications and devices connected to the Internet, the idea was to organize the IoT devices, using the distributed concept of micro-services, thus conceiving the concept of choreography applied to the micro-services, the structure IoT. Several tests, and researches in term of technologies available for this integration were carried out, and the possibility of the same was proven. This whole subject is treated gradually during three articles, and at the end of each, the results of the same are presented.

{\bf Key-words:} $<$IoT$>$,  $<$Internet$>$, $<$Micro$>$, $<$Interation$>$, $<$Services$>$.

\newpage

% %Lista de figuras%
% \renewcommand{\listfigurename}{\centering LISTA DE FIGURAS}
% \listoffigures
% \newpage

% %Lista de tabelas%
% \renewcommand{\listtablename}{\centering LISTA DE TABELAS}
% \listoftables
% \newpage

% %Lista de abreviaturas%
\section*{\centering{LISTA DE ABREVIATURAS}} 

SIGLA		– 	NOME COMPLETO 

LUMO		–	Laboratório de computação Móvel e Ubíqua

UbiComp	–  	Computação Ubíqua 

% \newpage

%Sumário%

\pagestyle{plain} %mostra numeração da página%
\tableofcontents


\newpage
\section{INTRODUÇÃO}


Com a evolução da computação distribuída  surgiu a necessidade de criação de novos paradigmas, dando assim, origem ao Micro-serviço. O termo “Arquitetura de Micro-serviços” surgiu nos últimos anos para descrever uma maneira específica de desenvolver suítes de serviços com implantação (deploy) independente. Esta arquitetura tem várias características-chave que reduzem a complexidade. Cada micro-serviço funciona como um processo separado. Consiste em interfaces impulsionadas por dados que normalmente têm menos de quatro entradas e saídas. Cada micro-serviço é auto-suficiente para ser implantado em qualquer lugar em uma rede, pois contém tudo o que é necessário para que ele funcione - bibliotecas, instalações de acesso a banco de dados e arquivos específicos do sistema operacional. Cada micro-serviço é construído em torno de uma única funcionalidade focada; Portanto, é mais eficaz. Desenvolvimento, extensibilidade, escalabilidade e integração são os principais benefícios oferecidos pela Arquitetura de Micro-serviços. 

Tem-se surgido muitos projetos utilizando este formato nos últimos anos e os resultados têm sido positivos, tanto que para muitos desenvolvedores o mesmo têm-se tornado a forma padrão de desenvolver aplicações. Entretanto, não existe muita informação que descreve o que são micro-serviços e como implementá-los (FOWLER et al., 2016). Utilizando-se da empresa Netflix como referência em micro-serviços, esta provê  muitos recursos gratuitos e de código aberto para desenvolvedores como Eureka, Hystrix, Ribbon entre outros. Estimativas apontam que a mesma faturou algo em torno de R\$ 1,1 bilhões somente no Brasil no ano de 2015 e fontes do mercado registraram que o canal de streaming faturou cerca de R\$ 260 milhões a mais do que a previsão mais otimista de faturamento do SBT no ano de 2015. (FELTRIN, 2016). A empresa Netflix é uma das pioneiras em micro-serviços, e este termo nem sequer existia quando o serviço por streaming da empresa começou a caminhar. Atualmente a plataforma da mesma é sustentada por um Gateway (Ponte de Ligação) de APIs que lida com cerca de dois bilhões de requisições todo o dia. No total as requisições citadas são tratadas por mais de 600 APIs (SMARTBEAR, 2016). 

Atualmente, um assunto também em discussão, que tem chamado a atenção desde pessoas com pouco conhecimento em tecnologia até pessoas que trabalham na área, é o IoT (Internet of Things) ou “Internet das Coisas”  que se refere a uma revolução tecnológica que tem como objetivo conectar itens utilizados no dia a dia à rede mundial de computadores. Cada dia surgem mais eletrodomésticos, meios de transporte e até mesmo acessórios vestíveis conectados à Internet e a outros dispositivos, como computadores e smartphones (ZAMBARDA, 2014). Segundo Ashton (Primeiro especialista a utilizar o termo “Internet das Coisas”) a limitação de tempo e da rotina fará com que as pessoas se conectem à Internet de outras maneiras, sendo para tarefas pessoais ou trabalho, permitindo o compartilhamento de informações e experiências existentes na sociedade. Segundo uma pesquisa realizada em 2015 pelo IDC (Corporação Internacional de dados), no mercado de IoT seria movimentado em 2016 cerca de US\$ 41 bilhões (IDC, 2016). 


Todas as evoluções tecnológicas na área de micro-serviços e IoT tem gerado grande interesse por parte dos desenvolvedores. Com base nas informações apresentadas, observa-se que são duas áreas distintas que crescem exponencialmente em razão do surgimento de novas tecnologias e têm-se necessidade de verificar relações que podem ser feitas entre as mesmas.  Ao construir estruturas de comunicação entre diferentes processos, é visto que, muitos produtos e abordagens enfatizam a inserção de inteligências significativas no próprio mecanismo de comunicação. Um exemplo do que foi citado é o Enterprise Service Bus (ESB), onde os os produtos do mesmo incluem recursos sofisticados para roteamento de mensagens, coreografia, transformação e aplicação de regras de negócios. As aplicações construídas a partir de micro-serviços visam ser independentes e coesas, e estes são coreografados utilizando protocolos RestFul (FOWLER et al., 2016). 

No ano de 1990 estava em alta uso a Arquitetura Orientada a Serviços (SOA). Foi um padrão que incluiu serviço como uma funcionalidade individual. O SOA trouxe muitas vantagens como velocidade, melhores fluxos de trabalho e vida útil mais longa das aplicações. Desta vez, foi do ponto de vista da criação de aplicativos desenvolvidos em torno de componentes de domínio de negócios e que poderiam ser desenvolvidos, manipulados e decompostos em serviços que se comunicassem por meio de APIs e protocolos de mensagens baseados em rede. Aqui é onde a Arquitetura de micro-serviços nasceu. A mesma adiciona agilidade, velocidade e eficiência quando se trata de implantação e modificação de sistemas. Como a tecnologia evolui, especificamente com IoT ganhando tanta tração, as expectativas das plataformas baseadas em nuvem mudaram. Big Data, termo que descreve imenso volumes de dado, se tornou um lugar comum e o mundo tecnológico começou a se mover para a economia de API. Este é o ponto onde o clássico SOA começou a mostrar problemas, demonstrando ser muito complicado, com centenas de interfaces e impossível definir granularidade.  (TAYAL, 2016)


Os micro-serviços hospedados em nuvem criaram um modelo de coleção de serviços, representando uma função específica. Os mesmos oferecem uma maneira de dimensionar a infra-estrutura tanto horizontal quanto verticalmente, proporcionando benefícios de longo prazo para as implantações de aplicações. Cada um dos serviços pode escalar com base nas necessidades. Dando o dinamismo das expectativas de implantação e escalabilidade que vem com o Micro-serviço, os mesmos precisam se tornar uma parte importante da estratégia IoT. (TAYAL, 2016)

Neste trabalho  tem-se por objetivo o desenvolvimento de uma interação entre as tecnologias citadas, através de uma interface de comunicação simples onde cada sistema embarcado se comunicará com algum micro-serviço genérico permitindo assim, a escalabilidade, sustentabilidade e independência dos serviços propostos. Será utilizado tecnologias como Spring Boot, uma plataforma Java criado por Rod Johnson baseado nos padrões de projeto inversão de controle (IoC) e injeção de dependência, Eureka, uma Interface de comunicação Java para micro-serviços para a construção dos Micro-serviços e a plataforma de prototipagem eletrônica NodeMcu ESP8266 para desenvolvimento do IoT que se comunicará com os mesmos. Para exemplo de aplicação, pode ser citado  um conjunto de dispositivos que iriam coletar informações de sensores e controladores, e torná-los visíveis na forma de dados. Os micro-serviços poderiam apenas processar esses dados e aplicar algumas regras a esses dados. Outros serviços também poderiam buscar dados de sistemas empresariais de terceiros, como sistemas CRM / ERP.



% \section{REVISÃO BIBLIOGRÁFICA}

\subsection{Micro-serviços e IoT}
%Lembre-se que as sessões e sub-sessões são determinadas por si para adequar-se ao seu trabalho.
Este trabalho tem por objetivo apresentar micro-serviços e IoT, e expor o desenvolvimento e integração dos mesmos.Os micro-serviços influenciam diretamente no modo em que são desenvolvidas e distribuídas as aplicações. Após estudos realizados nos últimos anos para descrever o termo “Arquitetura de Micro-serviços (Microservice Architecture)”, foi definido que, de uma maneira específica é possível desenvolver software como suítes de serviços com deploy (implantação) independente. Embora não exista uma definição precisa deste tipo de arquitetura, há certas características relacionadas à organização, à capacidade de negócios independentes, ao deploy automatizado, à inteligência e controle descentralizado de linguagens e de dados. (LEWIS, 2015).

Um exemplo de motivação para o uso de micro-serviços são os sistemas ERP (Enterprise Resource Planning ou Sistema para Planejamento de Recursos Empresariais), que são desenvolvidos basicamente para cuidar de toda a empresa, desde o financeiro, recursos humanos), produção, estoque, dentre outros. Em um Sistema para Planejamento de Recursos Empresariais todas as funcionalidades citadas são agrupadas dentro deste grande sistema, fazendo dela uma aplicação monolítica, ou seja, uma aplicação feita em somente uma unidade. Neste contexto, aplica-se também as vantagens e desvantagens dos sistemas monolíticos. Um dos principais pontos negativos é que se tem um grande ponto de falha, que significa que se houver algum erro no cadastro de produtos que deixa o sistema fora do ar, isto vai levar junto o sistema inteiro, incluindo funcionalidades não relacionadas com a mesma. Outro ponto negativo é a base de código, que se torna exponencialmente extensa de acordo com o tempo de desenvolvimento, tornando assim novos membros do projeto improdutivos por algum tempo, já que a complexidade do código é bem maior. (ALMEIDA, 2015). Em uma publicação feita por Sampaio (2015), o mesmo definiu através de estudos que o Micro-serviços são componentes de alta coesão, baixo acoplamento, autônomos e independentes, que representa um contexto de negócio de uma aplicação.

Um fato que ocorreu no ano de 2014 foi que o Docker, uma plataforma Open Source escrito em Go, que é uma linguagem de programação de alto desempenho desenvolvida dentro do Google (DIEDRICH, 2015), veio como um container portátil padronizado e está sendo muito utilizado pela comunidade. Uma razão importante para sua utilização generalizada que Adrian (Membro e fundador da eBay Research Labs)  observa é sua portabilidade e o aumento da velocidade com container que entregava algo em minutos ou horas e passou para segundos. Na figura \ref{fig:utilizacao-docker} é apresentado sua utilização entre os anos 2012 e 2016.



\begin{figure}[h]
\centering
\includegraphics[height=4.2cm]{imagens/docker}
\caption{Gráfico de utilzação do docker entre os anos 2012 e 2016.}
\label{fig:utilizacao-docker}
\end{figure}

A velocidade do micro-serviço permite e incentiva a implementação e estudo dos mesmos. Segundo Adrian (Membro e fundador da eBay Research Labs)  micro-serviços possui características comum, como: Implantação com pouca frequência, novas versões implantadas automaticamente, orquestração de uso geral não é necessário, uma vez que, sistemas inteiros são implantados com todas as partes ao mesmo tempo, arquiteturas utilizam centenas de micro-serviços e cada publicação é altamente customizada.
Seguindo adiante, o próximo passo que Adrian vê é orquestração para aplicações baseadas em padrões portáteis, em vez de dezenas de micro-serviços nas quais novas versões são automaticamente implantadas e que escalabilidade e disponibilidade são asseguradas, prevendo também um movimento contínuo de arquiteturas monolíticas para arquiteturas de micro-serviços. (STENBERG, 2015).

Implantar a Arquitetura de micro-serviços em empresas vai proporcionar diferentes benefícios para a estrutura de negócio como: usufruir de liberdade maior para o desenvolvimento de serviços de modo independente, implantar automaticamente através de ferramentas de integração contínua e código aberto, como Hudson, Jenkins e outras, possibilitar utilização de códigos escritos em linguagens diferentes para diferentes serviços utilizando comunicação REST através de Json ou XML, facilitar a ampliação e integração de micro-serviços com serviços terceirizados, através de APIs, organizar o código em função de capacidades de negócio, dando mais visão das ofertas e necessidades dos clientes. Dentre todos os benefícios citados é possível fazer o gerenciamento otimizado de falhas, o que significa que, se um serviço venha a falhar, os outros continuarão funcionando. Através dos micro-serviços, é possível identificar falhas com mais eficiência, visto que o particionamento favorece uma visão mais detalhada de cada serviço (PELOI, 2016). É possível observar na figura \ref{fig:art-scalability} as três dimensões da escalabilidade, onde o eixo X refere-se à escalabilidade horizontal, para ampliar a capacidade e disponibilidade da aplicação (cada servidor executa uma cópia idêntica do código), Z semelhante à do eixo X,  mas requer a presença de um componente que se responsabilize pelo roteamento das requisições ao servidor adequado, e o eixo Y que representa a terceira dimensão da escalabilidade, denominada decomposição funcional e é responsável por dividir a aplicação em uma série de serviços. A cada serviço corresponde um conjunto de funções (gerenciamento de pedidos, gerenciamento de clientes, entre outros).

\begin{figure}[h]
\centering
\includegraphics[height=6.2cm]{imagens/scalability}
\caption{The Art of Scalability - 2009.}
\label{fig:art-scalability}
\end{figure}


Segundo dados de Richardson(2014), diversas empresas estão utilizando micro-serviços, dentre as citadas estão: Comcast Cable, Uber, Netflix, Amazon, Ebay, SoundCloud, Karma, Groupon, Hailo, Gilt, Zalando, Lending Club, AutoScout24.
Os problemas associados ao desenvolvimento de software em larga escala ocorreram em torno da década de 1960. Na década de 1970 viu-se um enorme aumento de interesse da comunidade de pesquisa para o design de software em suas aplicações e  no processo de desenvolvimento. Nesta década o design foi muitas vezes considerado como uma atividade não associada com a implementação em si, e portanto requerendo um conjunto especial de notações e ferramentas. Por volta da década de 1980, a integração do design nos processos de desenvolvimento contribuiu para uma fusão parcial dessas duas atividades, tornando assim mais difícil fazer distinções puras.

As referências ao conceito de arquitetura de software também começaram a aparecer década de 1980. No entanto, uma base sólida sobre o tema foi estabelecida apenas em 1992 por Perry Wolf (autor do livro “Foundations for the study of software architecture"). Sua definição de arquitetura de software era distinta do design de software, e desde então tem-se gerado uma grande comunidade de pesquisadores estudando as aplicações práticas da arquitetura de software com base em micro-serviços, permitindo  assim que os conceitos sejam amplamente adotados pela indústria e pela academia.

O advento e a difusão da orientação por objetos, a partir dos anos 80 e, em particular, a década de 1990, trouxe sua própria contribuição para o campo da Arquitetura de Software. O clássico por Gamma et ai. abrange a concepção de software orientado a objetos e como traduzi-lo em código que apresenta uma coleção de soluções recorrentes, chamados padrões. Esta ideia não é nova nem exclusiva à Engenharia de Software, mas o livro é o primeiro compêndio a popularizar a idéia em grande escala. Na era pré-Gamma os padrões para soluções OO já estavam sendo utilizado: um exemplo típico de um padrão de projeto arquitetônico em programação orientada a objetos é o Model-View-Controller (MVC), que tem sido um dos insights seminais no desenvolvimento precoce de interfaces gráficas de usuário.(DRAGONI et al., 2016)

Cerca de sete anos atrás a empresa Netflix (provedora global de filmes e séries de televisão via streaming - distribuição de dados, geralmente de multimídia em uma rede através de pacotes) começou a migrar suas aplicações legadas para uma arquitetura baseada em APIs (Interface de programação de aplicativos) hospedadas na nuvem (local para armazenamento de dados online) da Amazon (empresa transnacional de comércio electrónico dos Estados Unidos com sede em Seattle), influenciando assim, o crescimento de uma ideologia na área de desenvolvimento de softwares que foi batizada pelo nome de “micro-serviço”.

Uma investigação realizada pela empresa Cisco (Companhia sediada em San José, Califórnia, Estados Unidos da América) em 2016 revela que, apesar de toda a euforia sobre a Internet das Coisas, o consumo de de vídeo via internet gera 63\% do tráfego global. A expectativa é que essa marca chegue a 79\% até 2020 e o tráfego de dados gerado por vídeos em resolução Ultra HD subirá de 1.6\% para 20.7\% do total em 2020. Um levantamento realizado pela Cisco VNI Mobile 2016 mostra que os dispositivos IoT mais simples geram uma quantidade de dados equivalentes a 7 veze o que é produzido por um celular comum (não um smartphone). Demandando pouco das redes de telecomunicações, os dispositivos IoT não representarão um grande pessoa para os provedores de infraestrutura na América Latina (IDC, 2016). 

Segundo o relatório “The State of Internet” de 2016, da Akamai (Empresa de Internet americana, sediada em Cambridge, Massachusetts), o país melhor colocado na faixa de redes com banda igual ou maior a 15 Mb/s é o Chile - 4,4\% de seus serviços de Internet atingem essa marca. Entretanto, para chegar a essa posição, o Chile investiu pesadamente entre 2014 e 2015, conseguindo crescer 150\% de um ano para outro. O Uruguai fica logo abaixo, com 4,1\% de sua Internet na faixa dos 15 Mb/s. Atualmente no Brasil, somente 1,1\% dos serviços atingem esta marca.

Na arquitetura de microserviços, se quisermos que um aplicativo seja colocado em esteróides, ele pode ser feito sem afetar outros serviços. Podemos apenas começar a executar este serviço específico em um hardware mais forte. Um microservice único pode ser atualizado nesta arquitetura, sem afetar outros ... a única condição é que o sistema de tempo de execução suporta isso. Cada microservice em uma plataforma pode ser desenvolvido em uma linguagem diferente - Java, C, C ++, Python, etc Governança granular é possível para cada microservice porque não tem dependência em outro. Ele pode ser monitorado e governado separadamente. Essa arquitetura descentraliza o gerenciamento de dados, uma vez que cada microserviço pode armazenar seus dados de uma maneira que se adapte a ele. Arquitetura Microservice suporta automação. É possível mover montagens inteiras de microservices de um ambiente de implementação para outro apenas usando as configurações de perfil com um único clique. Eles são muito mais resistentes do que as aplicações tradicionais. Isto é devido ao fato de que uma única aplicação pode ser retirada de um monte de aplicativos microservices, como estes são independentes uns dos outros.

A arquitetura do microservice tem suas vantagens óbvias e aquela é a razão porque assim que muitos negócios e serviços públicos proeminentes como Netflix, eBay, Amazon, o serviço digital do governo BRIT NICO, realestate.com.au, para diante, Twitter, Paypal, Gilt, Bluemix, Soundcloud , The Guardian, etc, apenas para citar alguns, todos se graduaram de arquitetura monolítica a microservices. Embora este seja o caso, assim como não há um plano perfeito, não há nenhuma arquitetura perfeita. O que funciona sob uma circunstância particular pode se tornar o gargalo em outro.

\subsection{Tecnologias}
Neste trabalho será utilizado no Back-end a tecnologia do netflix Service Discovery (Eureka), e para comunicar com este serviço será utilizado o circuito integrado Nodemcu Esp8266. Existem outras bibliotecas que podem trabalhar em conjunto com o Eureka que são: Circuit Breaker (Hystrix), Intelligent Routing (Zuul) and Client Side Load Balancing (Ribbon). 

\subsubsection{Zuul}
Zuul é a “porta da frente” para todas as requisições de dispositivos e sites para o back-end. O mesmo foi construído para permitir roteamento dinâmico, monitoramento, resiliência e segurança. O Zuul foi desenvolvido pela Netflix pelo fato de que o volume e a diversidade do tráfego da API do mesmo resultam em problemas de produção que surgem rapidamente e  sem aviso prévio, eles precisavam de um sistema que permita os mesmos mudar rapidamente o comportamento e reagir a estas situações.
O Zull utiliza uma variedade de diferentes tipos de filtros que permite-se aplicar rapidamente funcionalidades os serviços de ponta. Esses filtros ajudam a executar as seguintes funções: Autenticação e segurança, identificando requisitos de autenticação para cada recurso e rejeitando solicitações indesejadas. Insights e monitoramento, rastreamento de dados significativos e estatísticas, a fim de dar uma visão precisa da produção. Roteamento dinâmico, encaminhado dinamicamente solicitações para diferentes clusters de backend conforme necessário. Stress Testing, aumento gradual de tráfego para um cluster, a fim de avaliar o desempenho. Load Shedding, alocação de capacidade para cada tipo de solicitação e soltando pedidos que excedem o limite. Manipulação de resposta estática, construção de respostas diretamente na ponta ao invés de encaminhá-las para um cluster interno.
Dentre os vários componentes que integram a biblioteca do Zuul, estão: Zuul-core que contém funcionalidades a fim de compilar e executar filtros, Zuul-simple que mostra como construir um aplicativo com zuul-core, Zuul-netflix que adiciona componentes Netflix utilizando Ribbon para solicitações de roteamento. (Zuul, 2014)

\subsubsection{Ribbon}
Ribbon dá suporte à comunicação entre processos na nuvem e inclui balanceadores de carga desenvolvidos pela netflix. A tecnologia citada fornece os seguintes recursos: regras de balanceamento de carga múltiplas e conectáveis, integração com a descoberta de serviços, resiliência de falhas incorporada, clientes integrados com balanceadores de carga e configuração de clientes utilizando Archaius. O Ribbon é composto pelos seguintes projetos: Ribbon-core que inclui definições de interface e balanceamento de carga e cliente, implementações de balanceador de carga comuns, integração de cliente com balanceadores de carga e fábrica de clientes. Ribbon-eureka que inclui implementações do balanceador de carga com base no Eureka-client (bibliote para registro e descoberta de serviços). Ribbon-httpclient que inclui a inclui a implementação baseada em JSR-311 do cliente REST integrada com balanceadores de carga. (Ribbon, 2014)

\subsubsection{Hystrix}
Hystrix é um ambiente distribuído, inevitavelmente algumas das muitas dependências de serviços falharão, e esta biblioteca ajuda a controlar as interações entre serviços distribuídos adicionando tolerância de latência e lógica de tolerância a falhas. O mesmo faz isso isolando pontos de acesso entre os serviços, interrompendo falhas em cascata através deles, todas as quais melhoram a resiliência geral do sistema. Atualmente, dezenas de bilhões de threads isoladas e centenas de bilhões não isoladas são executadas utilizando o Hystrix todos os dias na Netflix. Isso resulta em uma melhoria dramática no tempo e atividade e resiliência. Hystrix é projeto para: proteger e controlar a latência e falhas de dependências acessadas por meio de bibliotecas de terceiros, interromper falhas em cascata em um sistema distribuído, recuperação rápida a falhas, monitoramento em tempo real, alertas e controle operacional. Quando se trata de micro-serviços, os mesmos contém dezenas de dependências com outros micro-serviços, o que ocasiona que se um deles falhar e o mesmo não estiver isolado destas falhas externas, corre o risco de também ser afetado. Por exemplo, um aplicativo que dependa de 40 serviços em que cada serviço tem 99,99\% de disponibilidade, pode se esperar: 99,99 \^ 40 = 99,6\% de tempo de atividade, 0.4\% de 1 bilhão de falhas resulta em 4 milhões de falhas. Mesmo que pequena a possibilidade de falha, se somar a quantidade de micro-serviços ao tempo de indisponibilidade que pode surgir por pequenas falhas, o problema pode ser facilmente escalável fazendo com que assim serviços importantes fiquem até mesmos horas indisponíveis. Quando toda a aplicação está funcionando e configurada de maneira correta, o fluxo de solicitações ocorrer conforme a figura \ref{fig:hystrix-overtime}.

\begin{figure}[h]
\centering
\includegraphics[height=6.2cm]{imagens/figura3}
\caption{Wiki Hystrix (Internet Overtime) - 2015).}
\label{fig:hystrix-overtime}
\end{figure}

Quando um dos muitos serviços se torna latente, ele pode bloquear toda a solicitação do usuário, conforme apresentado na figura \ref{fig:hystrix-dimensions-scaling}.

\begin{figure}[h]
\centering
\includegraphics[height=6.2cm]{imagens/figura4}
\caption{Wiki Hystrix (3 dimensions to scaling) - 2015).}
\label{fig:hystrix-dimensions-scaling}
\end{figure}

Com tráfego de alto volume, uma única dependência com latência excessiva, pode fazer com todos os recursos fiquem saturados em segundo. Cada ponto em um aplicativo que atinge a rede ou em uma biblioteca cliente que pode resultar em solicitações de rede é uma fonte de falha potencial. Pior que falhas, esses aplicativos também podem resultar em latências aumentadas entre os serviços, que faz backup de filas e outros recursos do sistema, causando ainda mais falhas em cascata em todo a aplicação, conforme a figura \ref{fig:hystrix-container}.

\begin{figure}[h]
\centering
\includegraphics[height=6.2cm]{imagens/figura5}
\caption{Wiki Hystrix (Container) - 2015).}
\label{fig:hystrix-container}
\end{figure}

A biblioteca Hystrix subjaz os seguintes princípios de design: Impedir que qualquer dependência única utilize todos as threads de usuários de container (como Tomcat), desperdiçando carga, fornecer soluções sempre que possível para proteger os usuários contra falhas, utilizando técnicas de isolamento para limitar o impacto de qualquer dependência, otimizar o tempo de descoberta através de métricas, monitoramento e alertas em tempo quase real, otimizando o tempo de recuperação por meio de propagação de baixa latência de alterações de configuração e suporte para alterações de propriedade dinâmicas na maioria dos aspectos do Hystrix, o que permite fazer modificações operacionais em tempo real com loops de realimentação de baixa latência, protegendo contra falhas em toda a execução do cliente de dependência, não apenas no tráfego da rede. (Hystrix, 2015)

\subsubsection{Eureka + Spring Cloud}
Spring Cloud fornece integrações Netflix OSS para Spring Boot por meio de auto-configuração e vinculação ao Spring Environment e outros padrões de programação Spring. Dentre os produtos spring Cloud encontra-se os clientes Eureka ou Service Discovery que é um dos princípios fundamentais de uma arquitetura baseada em micro-serviços. Configurar um micro-serviço é trabalhado pois envolve diversas técnicas de descoberta e registro de serviços, e com o Service Discovery da Netflix torna-se eficiente este trabalho pois com poucas anotações Java consegue-se criar uma aplicação simples Eureka. 

Eureka também vem com um componente de cliente baseado em Java, o cliente Eureka, que torna as interações com o serviço muito mais fácil. Ocliente também tem um balanceador de carga incorporado que faz balanceamento de carga round-robin (Algoritmos simples de agendamento e escalonamento de processos) básico. Quando um cliente se registra no mesmo, o mesmo fornece metadados sobre, como host e porta, dentre outras informações que podem ser encontradas na documentação. Se o registro falhar durante a configuração, a instância da aplicação é removida do registro. Em resumo, o Eureka é um serviço baseado em REST (Representational State Transfer) que é utilizado principalmente na AWS (Amazon Web Services) para localizar serviços com a finalidade de balanceamento de carga e failover (tolerância a falhas) de servidores de camada intermediária. 

A Amazon possui um produto chamado AWS ELB (Amazon Web Services Elastic Load Balancer), que é uma solução de balanceamento de carga para serviços de ponta expostos ao tráfego web do usuário final, e a diferença entre o mesmo e o produto da Netflix, é que o Eureka preenche a necessidade de balanceamento de carga médio. Embora teoricamente pode-se colocar serviços de nível intermediário atrás do AWS ELB, no EC2 classic (Elastic Compute Cloud) pode-se expor ao mundo exterior e perder toda a utilidade dos grupos de segurança AWS. O AWS ELB  também possui uma solução de balanceamento de carga em proxy (servidor intermediário para requisições entre cliente e servidor final) tradicional, enquanto no Eureka, o balanceamento ocorre no nível da instância, servidor e host. As instâncias do cliente sabem todas as informações sobre quais aplicações precisam conversar.

Uma questão a ser analisada no Eureka, é o DNS (Domain Name System). Dentre as aplicações existente para resolução de DNS está o Route 53, um serviço de nomeação, como o qual Eureka pode fornecer o mesmo para os servidores de nível médio. Route 53 é um serviço DNS e também pode fazer roteamento baseado em latência em regiões AWS. Eureka é análogo ao DNS interno e não tem nada a ver com os servidores DNS em todo o mundo. Eureka também é isolado no sentido de que não sabe sobre servidores em outras regiões AWS. Sua finalidade principal de manter informações é para balanceamento de carga dentro de uma região.

Na Netflix, além de desempenhar um papel crítico no balanceamento de carga de nível médio, o Eureka é utilizado para os seguintes fins: implementações com Netflix Asgard, um serviço para fazer atualizações de serviços de forma rápida e segura, registro e exclusão de instâncias e transporte de metadados específicos de aplicativos adicionais sobre serviços. Dentre os motivos para utilizar o Eureka está o fato que o mesmo provê uma solução para balanceamento de carga round-robin simples, e quem não estiver disposto a se registrar com o AWS ELB e expor seu tráfego externamente, o mesmo resolve este problema.

Com o Eureka, a comunicação é transparente, pois o mesmo fornece informações sobre os serviços desejados para comunicação, mas não impõe quaisquer restrições sobre o protocolo ou método de comunicação. Exemplificando, pode-se utilizar o Eureka para obter o endereço do servidor destino e utilizar protocolos como thrift, http (s) ou qualquer outro mecanismos RPC (Remote Procedure Call) que permite fazer conexões ou chamadas por espaço de endereçamento de rede. 

\subsubsection{Modelo Arquitetural Eureka}
O modelo arquitetural implantado na Netflix utilizando o Eureka é  descrita na figura \ref{fig:wiki-eureka-est}. Existe um cluster por região que conhece somente instâncias de sua região. Há pelo menos um servidor Eureka por zona para lidar com falhas da mesma. Os serviços se registram e, em seguida, a cada 30 segundos enviam os chamados “batimentos cardíacos” ou requisições para renovar seus registros. Se o cliente não renovar o registro, ele é retirado do servidor em cerca de 90 segundos. As informações de registro e renovações são replicadas para todos as conexões no cluster. Os clientes de qualquer zona podem procurar as informações do registro para localizar seus serviços que podem estar em qualquer zona e fazer chamadas remotas.

Para serviços não baseados em Java, tem-se a opção de implementar a parte do cliente utilizando o protocolo REST desenvolvido para o Eureka ou executar um “side car” que é uma aplicação Java com um cliente embutido Eureka que manipula os registros e conexões. Quando se trabalha com serviços em nuvem, pensar em resiliência se torna ímprobo. Eureka se beneficia dessa experiência adquirida, e é construído para lidar com falha de um ou mais servidores do mesmo.

\begin{figure}[h]
\centering
\includegraphics[height=6.2cm]{imagens/figura6}
\caption{Wiki Eureka - 2015).}
\label{fig:wiki-eureka-est}
\end{figure}

\subsubsection{NodeMCU}

Neste trabalho, será utilizado o NodeMCU para desenvolvimento das aplicaçoes IoT, pelo fato do mesmo conter um módulo WiFi, o que facilitará na comunicação via interface de rede com os micro-serviços. NodeMCU é um firmware baseado em eLua - uma implementação completa utilizando programação Lua para sistemas embarcados (ELUA, 2017). O mesmo foi projetado para ser integrado com o Chip WiFi ESP8266 desenvolvido pela empresa Espressif, situada em Shangai, especializada no ramo de IoT (SYSTEMS, 2017). O NodeMCU utiliza sistema de arquivos SPIFFS (SPI Flash File System) e seu repositório no Github consiste
 em 98.1\% de código na linguagem C - criada em 1972 por Dennis Ritchie (STEWART, 2015) e o demais existente em código escrito na linguagem Lua  - criada em 1993 por Roberto Ierusalimschy, Luiz Henrique de Figueiredo e Waldemar Celes (LUA, 2015).

 

\
































% Exemplo tabela
% \begin{table}[h]
% \centering
% \caption{ Modelo de como as tabelas devem ser inseridas no texto }
% \vspace{0.2in}
% \newcolumntype{C}{>{\centering\arraybackslash}X}%
% \newcommand{\rowstyle}[1]{%
%   \protected\gdef\currentrowstyle{#1}%
% }
% \begin{tabularx}{\textwidth}{>{\bf}C|C|C|C}
% \hline 
% \textbf {Índice} & \textbf{Coluna 01} &\textbf{ Coluna 02} & \textbf{Coluna 03} \\ \hline \hline
% Linha 01 & & & \\ \hline
% Linha 02 & & & \\ \hline                         

% \end{tabularx}
% \end{table}

% \section{DESENVOLVIMENTO}

\subsection{Introdução ao Eureka}

Neste trabalho tem-se por objetivo o a pesquisa e desenvolvimento de uma estrutura IoT baseado no padrão de micro-serviços, e para isto precisa-se utilizar um dos princípios que foi citada na revisão bibliográfica que é o "Service Discovery" ou descoberta de serviços. Como o projeto será baseado em Java, inicialmente será criado um projeto e  utilizará maven para gerenciamento de dependências e o Eureka Client para descoberta de serviços. Para poder utilizar o maven deve-se criar um arquivo de configurações no diretório raíz do projeto chamado "pom.xml", e a documentação completa pode se encontrar no site de seus desenvolvedor. Posteriormente será incluso o seguinte groupId org.springframework.cloud e o artifactId spring-cloud-starter-eureka, e para mais informações pode ser encontrada na documentação oficial do Spring Cloud Netflix.

Quando um cliente se registra com o Eureka, ele fornece meta-dados sobre si, indicador de estado ou saúde, página inicial, dentre outros. Eureka recebe mensagens heartbeat (disponibilidade) de cada instância pertencente a um serviço. Se algum heartbeat falhar, a instância é removido do registro.
Para inicializar um projeto com Eureka Client, será utilizado algumas anotações Java fornecidas pelo Eureka descritas a seguir: @Configuration, para utilizar recursos do projeto Spring Config para facilitar configurações de projetos Spring baseado em Java, @ComponentScan para buscar componentes em pacotes java, @EnableAutoConfiguration para ativar a ComponentScandescritas a seguir: @EnableEurekaClient para ativar a descoberta de serviços do Eureka, @RestController para criar um controlador Rest (Representational State Transfer), @RequestMapping para mapear as rotas da aplicação.

\begin{verbatim}
@Configuration
@ComponentScan
@EnableAutoConfiguration
@EnableEurekaClient
@RestController
public class Application {

    @RequestMapping("/")
    public String home() {
        return "Hello world";
    }

    public static void main(String[] args) {
        new SpringApplicationBuilder(Application.class)
          .web(true).run(args);
    }
}
\end{verbatim}

Para que possa surtir efeito na aplicação precisa fazer ajustes nas configurações do Eureka dentro do diretório resources da aplicação Java. Esta configuração é feita dentro de um arquivo Application.yml.

\begin{verbatim}
  Eureka
   cliente:
     ServiceUrl:
       DefaultZone: http: // localhost: 8761 / eureka / 
\end{verbatim}

Neste arquivo de configuração encontra-se uma peculariedade. O DefaultZone é a URL do serviço Eureka para qualquer cliente. O nome do aplicativo padrão (ID de serviço), o host e a porta podem ser acessadas respectivamente pelas variáveis de ambientes: \${spring.application.name} , \${spring.application.name} e \${server.port}.
A anotação Java @EnableEurekaClient faz com que o a aplicação corrente se registre no Eureka, para que assim possa localizar outros serviços.

\subsection{Status e Saúde do serviço}

Com a página de status e os indicadores de integridade de uma instância do Eureka é possível visualizar informações do serviço. Para acessar os indicadores de saúde deve-se configurar as rotas padrões de acesso a mesma. Por padrão, o eureka utiliza a conexão do cliente para determinar se um cliente está ativo. Caso não utilize o Discovery Client, não será propagado o status de verificação de integridade atual do serviço. Para funcionar corretamente os indicadores de saúde e status da aplicação, devem ser feitas as seguintes configurações:

\begin{verbatim}
eureka:
  instance:
    statusPageUrlPath: ${management.context-path}/info
    healthCheckUrlPath: ${management.context-path}/health
  client:
    healthcheck:
      enabled: true
\end{verbatim}

Para conseguir utilizar mais recursos e obter mais informações sobre o status da aplicação, a aplicação deve implementar seu próprio controle de integridade que se encontra no pacote com.netflix.appinfo.HealthCheckHandler

\subsection{Alterando o ID da instância Eureka}

Uma instancia registrada no Eureka possui seu ID, que identifica o serviço que está no mesmo. O Spring Cloud Eureka fornece o seguinte padrão de configuração: \$\{spring.cloud.client.hostname\}:\$\{spring.application.name\}:\$\{spring.application.instance\_id:
\$\{server.port\}\}. Como exemplo a URL fica da seguinte maneira: myhost:myapp:8080

\subsection{EurekaClient}

O próximo passo para aprender a utilizar os clientes Eureka, é aprende a utlizar o EurekaClient, que pode ser utilizado para descobrir instâncias do Eureka Server. Para fazer isto utilizando o framework desenvolvido para Java pela Spring Cloud, precisa primeiramente injetar a dependência do EurekaClient e criar um método que busque as instâncias registradas no Eureka.

\begin{verbatim}
@Autowired
private EurekaClient discoveryClient;

public String serviceUrl() {
    InstanceInfo instance = 
      discoveryClient.getNextServerFromEureka("STORES", false);
    return instance.getHomePageUrl();
}
\end{verbatim}

Não necessariamente precisa utilizar o EurekaClient. Também pode-se utilizar o DiscoveryClient. A diferença entre os dois, está na maneira de como é utilizado.

\begin{verbatim}
@Autowired
private DiscoveryClient discoveryClient;

public String serviceUrl() {
    List<ServiceInstance> list = 
      discoveryClient.getInstances("STORES");
    if (list != null && list.size() > 0 ) {
        return list.get(0).getUri();
    }
    return null;
}
\end{verbatim}

\subsection{Performanece de registro no Eureka}

Registrar um serviço no Eureka pode ser considerado um pouco lento, pelo fato de que ser uma instância também envolve um heartbeat periódico para o registro com duração padrão de 30 segundos. Um serviço não estará disponível para descoberta por clientes enquanto uma intância tenha todos os metadados em seu cache local. Para alterar o período em que isto ocorre pode ser configurado através da propriedade eureka.instance.leaseRenewalIntervalInSeconds. Entretanto, em produção, não deve ser alterado este padrão pelo fato de que, existem alguns cálculos internos do Eurela que fazem suposições de renovação de locação.

\subsection{Zonas Eureka}

Primeiramente, para se configurar uma zona Eureka, precisa-se ter certeza de que existem servidores Eureka implantados em cada zona e que eles são pares uns dos outros. Em seguida, precisa-se informar em qual zona o mesmo está. Para fazer isto será utilizado a propriedade metadataMap. E isto pode ser feito da seguinte forma:

\begin{verbatim}
eureka.instance.metadataMap.zone = zone1
eureka.client.preferSameZoneEureka = true
\end{verbatim}

\subsection{Primeiros passos com Eureka Server}

Primeiramente, para começar a utilizar o Eureka, como este projeto é baseado em Java no backend e utiliza maven como gerenciador de dependências, deve ser incluido nas mesmas a dependência com o groupId org.springframework.cloud e o artifactId spring-cloud-starter-eureka-server. Com esta dependência adicionada, será possível utilzar o Eureka Server.

Após adicionar esta dependência, deve ser criado a classe principal que se encarregará de iniciar a aplicação Eureka Server. Utilizando-se da anotação @EnableEurekaServer fornecida pelo framework e seguindo o padrão utilizado no Eureka Client para iniciar a aplicação, é possível ver um resultado. Um exemplo de código pode ser visto a seguir.

\begin{verbatim}
@SpringBootApplication
@EnableEurekaServer
public class Application {

    public static void main(String[] args) {
        new SpringApplicationBuilder(Application.class).web(true).run(args);
    }

}
\end{verbatim}

\subsection{Modo Autônomo}

A combinação entre o cliente e servidor Eureka e as pulsações para verificação de disponibilidade entre os mesmos, tornam o servidor Eureka Autônomo bastante resiliente à falha, contanto que haja algum tipo de monitoramento para mantê-lo funcionando. No modo autônomo, pode-se prefirir desativar o comportamento padrão do lado do cliente, para que ele não continue tentado alcançar seus pares caso haja falha. Para isto será feito diversas configurações como pode ser visto a seguir.


\begin{verbatim}
server:
  port: 8761

eureka:
  instance:
    hostname: localhost
  client:
    registerWithEureka: false
    fetchRegistry: false
    serviceUrl:
      defaultZone: http://${eureka.instance.hostname}:${server.port}/eureka/
\end{verbatim}

Com o Eureka, os registros podem ser ainda mais resistentes e disponíveis executando várias instâncias e pedindo-lhes para se registrarem uns com os outros. Tudo o que precisa para fazê-lo é configurar o serviceUrl dos pares.

\begin{verbatim}
---
spring:
  profiles: peer1
eureka:
  instance:
    hostname: peer1
  client:
    serviceUrl:
      defaultZone: http://peer2/eureka/

---
spring:
  profiles: peer2
eureka:
  instance:
    hostname: peer2
  client:
    serviceUrl:
      defaultZone: http://peer1/eureka/
\end{verbatim}

Neste exeplo, temos configurado, que o serviço pode ser utilizado para executar o mesmo servidor em 2 hosts (peer1 e peer2), executando-o em diferentes perfis Spring. Pode-se utilizar esta configuração para testar a descoberta dos pares em um único host, manipulando se em servidor Linux o arquivo hosts para resolver os nomes de host que pode ser encontrado dentro do diretório /etc/hosts.

Em alguns casos, é preferível que o Eureka utilize os endeços IP dos serviços ao invés do nome do host. Para isto deve ser definido a configuração eureka.instance.preferIpAddress como true e quando a aplicação se registrar com o Eureka, o mesmo utilizará seu endereço IP ao invés de seu nome de host.

\subsection{Clientes Hystrix}

Segundo Fowler (2016) é comum que os sistemas de software façam chamadas remotas para software em execução em diferentes processos, provavelmente em máquinas diferentes em uma rede. Uma das grandes diferenças entre chamadas em memória e chamadas remotas é que chamadas remotas podem falhar ou travar sem uma resposta até que algum limite de tempo limite seja atingido. O que é pior se você tem muitos chamadores em um fornecedor que não responde, então você pode ficar sem recursos críticos levando a falhas em cascata em vários sistemas. Em seu excelente livro Release It  , Michael Nygard popularizou o padrão de disjuntor para evitar este tipo de cascata catastrófica.

Com todos estes problemas que podem ocorrer na arquitetura de micro-serviços, a Netflix criou a biblioteca chamada Hystrix, que impelementa o padrão disjuntor que interromple automáticamente o serviço quando ocorre falhas. Uma falha de serviço no nível inferior de serviços pode causar falha em cascata em todo o caminho até o usuário. No Hystrix o padrão de limite de falhas são 20 em 5 segundos, e quando ocorre o circuito é aberto e a chamada não é feita, isto pode ser visto na figura \ref{fig:figura7}

\begin{figure}[h]
\centering
\includegraphics[height=4.2cm]{imagens/figura7}
\caption{Fallback Hystrix em falhas em cascata.}
\label{fig:figura7}
\end{figure}

\subsection{Primeiros passos com Hystrix}

Para utilizar o Hystrix, precisa seguir o mesmo padrão de quando se começa uma aplicação Eureka Server ou Client. No caso do padrão deste projeto backend baseado em Java com gerenciado de dependências maven, o que mudará será a anotação utilizada na classe principal do projeto e a dependência maven utilizada. Para incluir no projeto, será utilizado a dependência com o groupId org.springframework.cloud e o artifactId spring-cloud-starter-Hystrix, e a anotação @EnableCircuitBreaker na classe principal.

\begin{verbatim}
@SpringBootApplication
@EnableCircuitBreaker
public class Application {

    public static void main(String[] args) {
        new SpringApplicationBuilder(Application.class).web(true).run(args);
    }

}

@Component
public class StoreIntegration {

    @HystrixCommand(fallbackMethod = "defaultStores")
    public Object getStores(Map<String, Object> parameters) {
        //do stuff that might fail
    }

    public Object defaultStores(Map<String, Object> parameters) {
        return /* something useful */;
    }
}
\end{verbatim}

No exemplo de código acima, foi implementando uma classe que se contém um método que buscara os serviços, e para isto foi utilizado a anotação @HystrixCommand.

É possível ativar as métricas Hystrix e a central de gerenciamento do mesmo adicionando as dependências abaixo. O endpoint para acesso ao gerenciador é /hystrix.stream.

\begin{verbatim}
<dependency>
    <groupId>org.springframework.boot</groupId>
    <artifactId>spring-boot-starter-actuator</artifactId>
</dependency>
<dependency>
    <groupId>org.springframework.cloud</groupId>
    <artifactId>spring-cloud-starter-hystrix-dashboard</artifactId>
</dependency>
\end{verbatim}

\subsection{Ribbon}

Ribbon é um  balanceador de carga do lado do cliente que fornece controles sobre o comportamento dos clientes HTTP e TCP. Uma observação importante é que a anotação @FeignClient já utiliza Ribbon, fazendo com que assim seja desnecessário a utilização do Ribbon, entretanto, quando for preciso um controle mais versátil sobre a tecnologia é optável a utilização do mesmo.

Para incluir o ribbon no projeto, será utilizado o mesmo padrão de configurações feito anteriormente. O que será alterado é a o artifactId da dependência maven que agora será utilizado spring-cloud-starter-ribbon, e para configurar o cliente Ribbon criado uma classe de configuraçao e será anotado com @RibbonClient. 

\begin{verbatim}
@Configuration
@RibbonClient(name = "foo", configuration = FooConfiguration.class)
public class TestConfiguration {
}
\end{verbatim}

\subsection{FeignClient}

FeignClient é uma biblioteca que faz com que clientes de serviços web sejam escritos de forma mais fácil. Para utilizá-lo é preciso instalar a dependência spring-cloud-starter-feign e anotar a classe principal com @EnableFeignClients. O mesmo provê suporte para anotações Spring MVC e por utilizar o mesmo conversor de mensagens HTTP que o Spring Web, é integra com Hystrix para fornecer um cliente com balanceamento de carga.

\begin{verbatim}
@Configuration
@ComponentScan
@EnableAutoConfiguration
@EnableEurekaClient
@EnableFeignClients
public class Application {

    public static void main(String[] args) {
        SpringApplication.run(Application.class, args);
    }

}
\end{verbatim}

Ao anotar uma interface com @FeignClient, pode ser mapeado os métodos para que consiga acesso a endpoints da biblioteca. O nome do método será qualificado e aplicado ao contexto da aplicação, fazendo com que assim não seja implementado corpo ao método pois será apenas repassados chamadas REST.

\begin{verbatim}
@FeignClient("stores")
public interface StoreClient {
    @RequestMapping(method = RequestMethod.GET, value = "/stores")
    List<Store> getStores();

    @RequestMapping(method = RequestMethod.POST, value = "/stores/{storeId}", consumes = "application/json")
    Store update(@PathVariable("storeId") Long storeId, Store store);
}
\end{verbatim}

Cada Cliente Feign faz parte de um conjunto de componentes que trabalham juntos para comunicar-se via HTTP. O Spring Cloud permite com que se tenha controle total sobre clientes Feign declarando uma classe de configuração que implemente determinados métodos do FeignClient. Duas das possíveis configuração, sao: modificar o padrão de Contrato que o FeignClient utiliza para que assim seja personalizado o padrão de comunicação REST que o mesmo utiliza e modificar o método de autenticação do FeignClient.

\begin{verbatim}
@Configuration
public class FooConfiguration {
    @Bean
    public Contract feignContract() {
        return new feign.Contract.Default();
    }

    @Bean
    public BasicAuthRequestInterceptor basicAuthRequestInterceptor() {
        return new BasicAuthRequestInterceptor("user", "password");
    }
}
\end{verbatim}

Em alguns casos, pode ser necessário outros métodos mais especificos para que seja personalizado as configurações. Neste caso, pode-se utilizar a API Feign Builder. Abaixo está um exemplo da utilização do mesmo.

\begin{verbatim}
@Import(FeignClientsConfiguration.class)
class FooController {

	private FooClient fooClient;

	private FooClient adminClient;

    @Autowired
	public FooController(
			Decoder decoder, Encoder encoder, Client client) {
		this.fooClient = Feign.builder().client(client)
				.encoder(encoder)
				.decoder(decoder)
				.requestInterceptor(new BasicAuthRequestInterceptor("user", "user"))
				.target(FooClient.class, "http://PROD-SVC");
		this.adminClient = Feign.builder().client(client)
				.encoder(encoder)
				.decoder(decoder)
				.requestInterceptor(new BasicAuthRequestInterceptor("admin", "admin"))
				.target(FooClient.class, "http://PROD-SVC");
    }
}
\end{verbatim}

\subsection{Zuul}



\newpage

% \setlength{\voffset}{-2.54cm}
% \setlength{\hoffset}{-2.54cm}
\addtocounter{section}{1}
\addcontentsline{toc}{section}{\arabic{section} \quad  ARTIGOS}
\includepdf[pages=-, addtotoc={
    1,subsection,1, \quad IoT - A Internet das coisas,p1
    }]{artigos/artigo1/bare_jrnl.pdf}
\includepdf[pages=-, addtotoc={
    1,subsection,1, \quad Micro-serviços,p1
    }]{artigos/artigo2/bare_jrnl.pdf} 
\includepdf[pages=-, addtotoc={
    1,subsection,1, \quad Integração entre IoT e Micro-serviços,p1
    }]{artigos/artigo3/bare_jrnl.pdf}  
% \section[asdasd]{}
% \addcontentsline{toc}{section}{IoT - A Internet das coisas}
% \includepdf[pages=-, offset=75 -75]{artigos/artigo1/bare_jrnl.pdf}
% \addcontentsline{toc}{section}{Micro-serviços}
% \includepdf[pages=-, offset=75 -75]{artigos/artigo2/bare_jrnl.pdf}
% \addcontentsline{toc}{section}{IoT e integração com Micro-serviços}
% \includepdf[pages=-, offset=75 -75]{artigos/artigo3/bare_jrnl.pdf}

\setlength{\voffset}{0cm}
\setlength{\hoffset}{0cm}


% \input{4_resultados}
\section{CONCLUSÕES E TRABALHOS FUTUROS}
Durante este trabalho foram desenvolvidos três artigos, nos quais contêm diversas informações sobre IoT e Micro-serviços. Dentres estes, foram desenvolvidos testes e exemplos práticos dos mesmos, incluindo a integração deles. Como é necessário conexão com internet para compartilhamento de informações, foi utilizado o dispositivo NodeMCU ESP8266, pois este contém módulo WiFi integrado, o que possibilita a comunicação com outros serviços e dispositivos, além de um microcontrolador para pequeno processamento. Também foram utilizadas ferramentas prontas para desenvolvimento dos serviços, o que facilita e agiliza o desenvolvimento da aplicação. Considerando a arquitetura que foi estabelecida, conclui-se que é possível desenvolver esta integração, sendo assim, o objetivo inicial do trabalho é concluido. A mesma abre possibilidades para a integração entre novas tecnologias e serviços, pois visa a independência e alta coesão de cada serviço e dispositivo. Automação residencial e industrial, são apenas duas das áreas que podem ser beneficidadas, devido a simples configuração que é empregada a esta estrutura.
%%%%%%%%%%%%%%%%%%%%%%%%%%%%% Referências %%%%%%%%%%%%%%%%%%%%%%%%%%%%%
\renewcommand{\refname}{\centering REFERÊNCIAS} %Centraliza nome Referencias%
\addcontentsline{toc}{section}{REFERÊNCIAS} %adiciona referencias ao sumario
\nocite{*}
\bibliographystyle{acm}
\makeatletter
\renewcommand\@biblabel[1]{}
\makeatother
\bibliography{references}

%%%%%%%%%%%%%%%%%%%%%%%%%%%%%%%%%%%%%%%%%%%%%%%%%%%%%%%%%%%%%%%%%%%%%%%

% %%%%%%%%%%%%%%%%%%%%%%%%%%%%% ANEXO %%%%%%%%%%%%%%%%%%%%%%%%%%%%%
\section*{\centering{A – ANEXOS E APÊNDICES 1}}
\addcontentsline{toc}{section}{A - ANEXOS E APÊNDICES 1}

Anexos e apêndices são materiais adicionais, utilizados para complementar o texto, acrescentados ao final do trabalho, com a finalidade de esclarecimento ou de comprovação.

Apêndices são elaborados pelo autor e visam complementar uma argumentação. Os Anexos não são elaborados diretamente pelo autor e servem de fundamentação teórica, comprovação e ilustração (ex. mapas, leis, estatutos entre outros). Os apêndices devem aparecer antes dos anexos.



\end{document}