\section*{\centering{RESUMO}}
% Um resumo de trabalho de conclusão de curso é do tipo informativo e deve conter somente um parágrafo. A estrutura do resumo deve conter essencialmente os seguintes tópicos: apresentar inicialmente os objetivos do trabalho (o que foi feito?), a justificativa (porquê foi feito) e, finalmente, os resultados alcançados. O resumo deve informar ao leitor todas as informações importantes para o que o leitor possa entender o trabalho desenvolvido, quais foram as finalidades, a metodologia que o autor utilizou e os resultados obtidos. Deve conter frases curtas, porém completas (evitar estilo telegráfico); usar o tempo verbal no passado para os principais resultados e presente para comentários ou para salientar implicações significativas.  O resumo em português e inglês são obrigatórios e não devem passar de 200 palavras.

Este trabalho tem por objetivo a apresentação dos conceitos IoT \emph{(Internet of Things)} e Micro-serviços, e provar que é possível desenvolver uma estrutura, que possibilita a integração entre estas tecnologias. Uma das justificativas para o desenvolvimento deste trabalho, tende para o fato que atualmente tem surgido diversos estudos sobre os assuntos citados, e têm beneficiado as aplicações desenvolvidas, com alta coesão e baixo acoplamento quando se trata de micro-serviços, e propiciado o compartilhamento de informações e integração a rede mundial de Internet quando se trata de IoT. Associando estas duas tecnologias que são direcionadas a aplicações e dispositivos conectados a internet, teve-se a idéia de organizar os dispositivos IoT, utilizando o conceito distribuido dos micro-serviços, concebendo assim, o conceito de coreografia aplicado aos micro-serviços, a estrutura IoT. Foram realizados diversos testes, e pesquisas em termo de tecnologias disponíveis para esta integração, e foi comprovado a possibilidade da mesma. Todo este assunto é tratado gradativamente durante três artigos, e ao final de cada um, é apresentado os resultados do mesmo.

{\bf Palavras-chave:} $<$IoT$>$,  $<$Internet$>$, $<$Micro$>$, $<$Interação$>$, $<$serviços$>$.