\section{INTRODUÇÃO}

IoT e Micro-serviços, dois assuntos distintos, mas que de certa forma pode haver uma conexão entre os mesmos. Os dois assuntos foram explanados durante dois artigos, primeiramente IoT - A Internet das coisas e posteriormente Micro-serviços de Marques (2017). Neste contexto, aplica-se a integração dos mesmos, pois de fato, como são assuntos que que vem sendo estudados nestes artigos, obteve-se a idéia de fazer com que os dois trabalhem em conjunto. Micro-serviço é um padrão têm originado muitos projetos, e os resultado têm sido positivos. Segundo Fowler (2014), Micro-serviço é mais um novo termo na área de arquitetura de software que descreve um estilo de sistemas de software, que tem se tornando o estilo padrão para o desenvolvimento de aplicativos corporativos. Algumas características como alta coesão, autonomia, resiliência, observáveis, automatização e centralização no domínio de negócio fazem parte da arquitetura de micro-serviços. 

Em um podcast realizado pela empresa Hipsters.tech que faz publicação de podcasts sobre tecnologias, no qual se encontrava funcionários da empresa Netflix, dentre eles Fabio Kung (senior software engineer), cita como a empresa está crescendo, e que um dos objetivos da mesma é ter uma estabilidade mais palatável. O mesmo também fala que atualmente, ainda grande parte dos sistemas da empresa, funciona de forma molítica, e estão trabalhando no desacoplamento dos mesmos, para que um não afete os outros, e tenha a possibilidade de escalar facilmente partes específicas do sistema. Estimativas apontam que a empresa Netflix têm faturado somente no Brasil no ano de 2015 algo em torno de R\$ 260 milhões com streaming, e para gerar tal tráfego de streaming de dados, precisa-se de uma arquitetura robusta, para que atenda o mesmo (FELTRIN, 2016).

Outra empresa que tem trabalhado com micro-serviços é a Amazon, uma das primeiras empresas em que migrou suas aplicações de um enorme sistema monolítico, para uma estrutura de micro-serviços, à procura de um modelo mais perspicaz quando se trata de atualização e suporte a aproximadamente 2 milhões de solicitações de 800 tipos diferentes de dispositivos. Grandes empresas atuais estão a desmontar os modelos arquiteturais monolíticos, privilegiando componentes mais pequenos e independentes que trabalham em conjuento para resolver determinados problemas (WORLD, 2016).

Aproveitando-se deste contexto tecnológico de distribuição de dados, um assunto que também está em dicussão é o IoT. O mesmo refere-se a uma revolução tecnológica que tem como objetivo, conectar itens utilizados no dia a dia à rede mundial de computadores. Segundo uma pesquisa realizada pelo IDC (Corporação Internacional de dados), em 2016 foi movimentado em média de US\$41 bilhões somente nesta área.

O objetivo deste trabalho é desenvolver três artigos, o primeiro sobre IoT, o segundo sobre micro-serviços e o terceiro sobre integração entre IoT e micro-serviços. Considerando isto, será desenvolvido os mesmos, e cada artigo estará organizado da seguinte forma: uma introdução ao assunto, uma revisão bibliográfica sobre o mesmo, a parte de desenvolvido de cada um e finalmente conclusão individual dos mesmos. Ao encerrar a escrita dos três artigos, será feito uma conclusão geral do trabalho.

