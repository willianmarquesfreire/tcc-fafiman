\section{INTRODUÇÃO}


Com a evolução da computação distribuída  surgiu a necessidade de criação de novos paradigmas, dando assim, origem ao Micro-serviço. O termo “Arquitetura de Micro-serviços” surgiu nos últimos anos para descrever uma maneira específica de desenvolver suítes de serviços com implantação (deploy) independente. Esta arquitetura tem várias características-chave que reduzem a complexidade. Cada micro-serviço funciona como um processo separado. Consiste em interfaces impulsionadas por dados que normalmente têm menos de quatro entradas e saídas. Cada micro-serviço é auto-suficiente para ser implantado em qualquer lugar em uma rede, pois contém tudo o que é necessário para que ele funcione - bibliotecas, instalações de acesso a banco de dados e arquivos específicos do sistema operacional. Cada micro-serviço é construído em torno de uma única funcionalidade focada; Portanto, é mais eficaz. Desenvolvimento, extensibilidade, escalabilidade e integração são os principais benefícios oferecidos pela Arquitetura de Micro-serviços. 

Tem-se surgido muitos projetos utilizando este formato nos últimos anos e os resultados têm sido positivos, tanto que para muitos desenvolvedores o mesmo têm-se tornado a forma padrão de desenvolver aplicações. Entretanto, não existe muita informação que descreve o que são micro-serviços e como implementá-los (FOWLER et al., 2016). Utilizando-se da empresa Netflix como referência em micro-serviços, esta provê  muitos recursos gratuitos e de código aberto para desenvolvedores como Eureka, Hystrix, Ribbon entre outros. Estimativas apontam que a mesma faturou algo em torno de R\$ 1,1 bilhões somente no Brasil no ano de 2015 e fontes do mercado registraram que o canal de streaming faturou cerca de R\$ 260 milhões a mais do que a previsão mais otimista de faturamento do SBT no ano de 2015. (FELTRIN, 2016). A empresa Netflix é uma das pioneiras em micro-serviços, e este termo nem sequer existia quando o serviço por streaming da empresa começou a caminhar. Atualmente a plataforma da mesma é sustentada por um Gateway (Ponte de Ligação) de APIs que lida com cerca de dois bilhões de requisições todo o dia. No total as requisições citadas são tratadas por mais de 600 APIs (SMARTBEAR, 2016). 

Atualmente, um assunto também em discussão, que tem chamado a atenção desde pessoas com pouco conhecimento em tecnologia até pessoas que trabalham na área, é o IoT (Internet of Things) ou “Internet das Coisas”  que se refere a uma revolução tecnológica que tem como objetivo conectar itens utilizados no dia a dia à rede mundial de computadores. Cada dia surgem mais eletrodomésticos, meios de transporte e até mesmo acessórios vestíveis conectados à Internet e a outros dispositivos, como computadores e smartphones (ZAMBARDA, 2014). Segundo Ashton (Primeiro especialista a utilizar o termo “Internet das Coisas”) a limitação de tempo e da rotina fará com que as pessoas se conectem à Internet de outras maneiras, sendo para tarefas pessoais ou trabalho, permitindo o compartilhamento de informações e experiências existentes na sociedade. Segundo uma pesquisa realizada em 2015 pelo IDC (Corporação Internacional de dados), no mercado de IoT seria movimentado em 2016 cerca de US\$ 41 bilhões (IDC, 2016). 


Todas as evoluções tecnológicas na área de micro-serviços e IoT tem gerado grande interesse por parte dos desenvolvedores. Com base nas informações apresentadas, observa-se que são duas áreas distintas que crescem exponencialmente em razão do surgimento de novas tecnologias e têm-se necessidade de verificar relações que podem ser feitas entre as mesmas.  Ao construir estruturas de comunicação entre diferentes processos, é visto que, muitos produtos e abordagens enfatizam a inserção de inteligências significativas no próprio mecanismo de comunicação. Um exemplo do que foi citado é o Enterprise Service Bus (ESB), onde os os produtos do mesmo incluem recursos sofisticados para roteamento de mensagens, coreografia, transformação e aplicação de regras de negócios. As aplicações construídas a partir de micro-serviços visam ser independentes e coesas, e estes são coreografados utilizando protocolos RestFul (FOWLER et al., 2016). 

No ano de 1990 estava em alta uso a Arquitetura Orientada a Serviços (SOA). Foi um padrão que incluiu serviço como uma funcionalidade individual. O SOA trouxe muitas vantagens como velocidade, melhores fluxos de trabalho e vida útil mais longa das aplicações. Desta vez, foi do ponto de vista da criação de aplicativos desenvolvidos em torno de componentes de domínio de negócios e que poderiam ser desenvolvidos, manipulados e decompostos em serviços que se comunicassem por meio de APIs e protocolos de mensagens baseados em rede. Aqui é onde a Arquitetura de micro-serviços nasceu. A mesma adiciona agilidade, velocidade e eficiência quando se trata de implantação e modificação de sistemas. Como a tecnologia evolui, especificamente com IoT ganhando tanta tração, as expectativas das plataformas baseadas em nuvem mudaram. Big Data, termo que descreve imenso volumes de dado, se tornou um lugar comum e o mundo tecnológico começou a se mover para a economia de API. Este é o ponto onde o clássico SOA começou a mostrar problemas, demonstrando ser muito complicado, com centenas de interfaces e impossível definir granularidade.  (TAYAL, 2016)


Os micro-serviços hospedados em nuvem criaram um modelo de coleção de serviços, representando uma função específica. Os mesmos oferecem uma maneira de dimensionar a infra-estrutura tanto horizontal quanto verticalmente, proporcionando benefícios de longo prazo para as implantações de aplicações. Cada um dos serviços pode escalar com base nas necessidades. Dando o dinamismo das expectativas de implantação e escalabilidade que vem com o Micro-serviço, os mesmos precisam se tornar uma parte importante da estratégia IoT. (TAYAL, 2016)

Neste trabalho  tem-se por objetivo o desenvolvimento de uma interação entre as tecnologias citadas, através de uma interface de comunicação simples onde cada sistema embarcado se comunicará com algum micro-serviço genérico permitindo assim, a escalabilidade, sustentabilidade e independência dos serviços propostos. Será utilizado tecnologias como Spring Boot, uma plataforma Java criado por Rod Johnson baseado nos padrões de projeto inversão de controle (IoC) e injeção de dependência, Eureka, uma Interface de comunicação Java para micro-serviços para a construção dos Micro-serviços e a plataforma de prototipagem eletrônica NodeMcu ESP8266 para desenvolvimento do IoT que se comunicará com os mesmos. Para exemplo de aplicação, pode ser citado  um conjunto de dispositivos que iriam coletar informações de sensores e controladores, e torná-los visíveis na forma de dados. Os micro-serviços poderiam apenas processar esses dados e aplicar algumas regras a esses dados. Outros serviços também poderiam buscar dados de sistemas empresariais de terceiros, como sistemas CRM / ERP.


