\section{INTRODUÇÃO}

IoT e Micro-serviços, dois assuntos distintos, mas que de certa forma pode haver uma conexão entre os mesmos. Os dois assuntos foram explanados durante dois artigos, primeiramente IoT - A Internet das coisas e posteriormente Micro-serviços de Marques (2017). Neste contexto, aplica-se a integração dos mesmos, pois de fato, como são assuntos que que vem sendo estudados nestes artigos, obteve-se a idéia de fazer com que os dois trabalhem em conjunto. Micro-serviço é um padrão que segundo Fowler (2016) têm originado muitos projetos, e os resultado têm sido positivos, entretanto, não existe muita informação que descreve o que são e como implementá-los.

Em um podcast realizado pela empresa Hipsters.tech que faz publicação de podcasts sobre tecnologias, no qual se encontrava funcionários da empresa Netflix, dentre eles Fabio Kung (senior software engineer), cita que como e empresa está crescendo, um dos objetivos é que a mesma tenha uma estabilidade mais palatável. O mesmo também fala que atualmente, ainda grande parte dos sistemas da empresa, funciona de forma molítica, e estão trabalhando no desacoplamento dos mesmos, para que um não afete os outros, ter a possibilidade de escalar facilmente partes específicas do sistema. Estimativas apontam que a empresa Netflix têm faturado somente no Brasil no ano de 2015 algo em torno de R\$ 260 milhões, e para gerar tal tráfego de streaming de dados, precisa-se de uma arquitetura robusta, para que atenda o mesmo (FELTRIN, 2016).

A Amazon foi uma das primeiras empresas, em que migrou suas aplicações de um enorme sistema monolítico, para uma estrutura de micro-serviços, à procura de um modelo mais perspicaz quando se trata de atualização e suporte a aproximadamente 2 milhões de solicitações de 800 tipos diferentes de dispositivos. Grandes empresas atuais estão a desmontar os modelos arquiteturais monolíticos, privilegiando componentes mais pequenos e independentes que trabalham em conjuento para resolver determinados problemas (WORLD, 2016).

Segundo Fowler (2014), Micro-serviço é mais um novo termo na área de arquitetura de software. Segundo ele, a terminologia descreve um estilo de sistemas de software, que tem se tornando o estilo padrão para o desenvolvimento de aplicativos corporativos. Algumas características como alta coesão, autonomia, resiliência, observáveis, automatização e centralização no domínio de negócio fazem parte da arquitetura de micro-serviços. 

Aproveitando-se deste contexto tecnologico de distribuição de dados, um assunto que está em dicussão até mesmo entre os leigos, é o IoT. O mesmo refere-se a uma revolução tecnológica que tem como objetivo, conectar itens utilizados no dia a dia à rede mundial de computadores. Segundo uma pesquisa realizada pelo IDC (Corporação Internacional de dados), em 2016 seria movimentado em média de US\$41 bilhões somente na área de IOT (IDC, 2016).

