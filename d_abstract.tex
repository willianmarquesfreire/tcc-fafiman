%Abstract%
\section*{\centering{ABSTRACT}} 
% $<$Resumo em Inglês - Write here the abstract of your work$>$
This paper aims to present the IoT \emph{(Internet of Things)} concepts and Micro-services, and to prove that it is possible to develop a structure that allows the integration between these technologies. One of the justifications for the development of this work, tends to the fact that several studies have already appeared on the mentioned subjects, and have benefited the applications developed, with high cohesion and low coupling when it comes to micro-services, and favoring the sharing of Information and integration into the global Internet network when it comes to IoT. Associating these two technologies that are directed to applications and devices connected to the Internet, the idea was to organize the IoT devices, using the distributed concept of micro-services, thus conceiving the concept of choreography applied to the micro-services, the structure IoT. Several tests, and researches in term of technologies available for this integration were carried out, and the possibility of the same was proven. This whole subject is treated gradually during three articles, and at the end of each, the results of the same are presented.

{\bf Key-words:} $<$IoT$>$,  $<$Internet$>$, $<$Micro$>$, $<$Interation$>$, $<$Services$>$.