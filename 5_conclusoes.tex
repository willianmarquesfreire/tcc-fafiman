\section{CONCLUSÕES E TRABALHOS FUTUROS}
Durante este trabalho foram desenvolvidos três artigos, nos quais contêm diversas informações sobre IoT e Micro-serviços. Dentres estes, foram desenvolvidos testes e exemplos práticos dos mesmos, incluindo a integração deles. Como é necessário conexão com internet para compartilhamento de informações, foi utilizado o dispositivo NodeMCU ESP8266, pois este contém módulo WiFi integrado, o que possibilita a comunicação com outros serviços e dispositivos, além de um microcontrolador para pequeno processamento. Também foram utilizadas ferramentas prontas para desenvolvimento dos serviços, o que facilita e agiliza o desenvolvimento da aplicação. Considerando a arquitetura que foi estabelecida, conclui-se que é possível desenvolver esta integração, sendo assim, o objetivo inicial do trabalho é concluido. A mesma abre possibilidades para a integração entre novas tecnologias e serviços, pois visa a independência e alta coesão de cada serviço e dispositivo. Automação residencial e industrial, são apenas duas das áreas que podem ser beneficidadas, devido a simples configuração que é empregada a esta estrutura.